\chapter{Conclusiones y trabajo futuro}

Exponemos algunas ideas derivadas de nuestro trabajo. A partir del
fichero \texttt{costes.xlsx} obtenemos un desglose de tiempo
dedicado al proyecto. Dedicamos 24,5 horas a la implementación,
10 horas al análisis del problema y 72,5 horas a la memoria.

\section{Objetivos alcanzados}

Entregamos una solución software libre capaz de almacenar
y recuperar registros de trabajo de un paraguas vibrador sobre
cualquier plataforma que hile nuestras librerías e implemente
sus dependencias. Además, ofrecemos un controlador para el
integrado FRAM de Fujitsu \texttt{MB85RS4MT} \cite{Fujitsu:MB85RS4MT}.

Gracias al cuidadoso análisis del problema antes de decidir
soluciones reducimos la complejidad y costes del proyecto.

Junto al software entregamos esta memoria, en la que hemos
descrito una metodología de desarrollo software apropiada
para este tipo proyectos.

\section{Coste final}

Comparamos el coste que estimamos en el capítulo anterior
y los costes que finalmente tenemos. Como ya sabemos, esta estimación
la hacemos con los dos años en datos que tengo almacenados acerca de mis
tiempos de desarrollo y costes en mi empresa.

\subsection{Tiempo}

Recuperamos la tabla que hicimos para el capítulo de implementación y la
modificamos para contrastar los datos estimados con los datos que hemos
extraído en el desarrollo del proyecto.

Se puede consultar con detalle el tiempo dedicado al proyecto en el
documento Excel \texttt{costes.xlsx}, incluido en el repositorio.

\begin{table}[!h]
    \centering
    \begin{tabular}{|l|c|c|c|}
        \hline
        \multicolumn{1}{|c|}{\textbf{Nombre}}                                                                                               & \textbf{Horas estimadas}   & \textbf{Días estimados} & \textbf{Horas reales} \\ \hline \hline
        \begin{tabular}[c]{@{}l@{}}Modelado de las reglas de\\ procesamiento de eventos\\ (registro)\end{tabular}                  & 8$\sim$11,5 horas & 2$\sim$3 días  & 4                                                                                                  \\ \hline
        \begin{tabular}[c]{@{}l@{}}Modelado de las reglas de\\ almacenamiento y recuperación\\ de eventos (registros)\end{tabular} & 8$\sim$15,5 horas & 2$\sim$3 días  & 5                                                                                                  \\ \hline
        Driver FRAM                                                                                                                & 3$\sim$6 horas    & 1$\sim$2 días  & 9,25                                                                                                  \\ \hline
        Integración con HMI                                                                                                        & 1$\sim$9 horas    & 1$\sim$2 días  & 5,25                                                                                                 \\ \hline
    \end{tabular}
    \caption{\textit{Al hacer el análisis nos estamos dando cuenta que olvidamos desglosar y estimar el diseño de los gráficos de la interfaz gráfica.
    A este diseño le dedicamos una hora.}}
\end{table}

Según nuestra estimación de tiempo de desarrollo, la cual no tenía en cuenta el tiempo de analizar
el problema puesto que sin el análisis del problema no tenemos un desglose sobre el que poder hacer
una estimación, en el mejor de los casos tardábamos en desarrollar la solución 20 horas, mientras
que en el peor de los casos 42 horas. 

En analizar el problema, contando el tiempo de redactar la memoria,
hemos tardado 10 horas.

Según los datos que registramos \textbf{tardamos en la implementación 24,5 horas distribuidas a lo largo de
10 días}.

\subsection{Dinero}

Nuestro coste de desarrollo depende enteramente del importe que
facturamos a la hora. El desglose de los costes que consideramos
en este importe que facturamos los exponemos en la
\hyperref[sec:precio_hora]{sección 2.10.1}.

Nuestro desarrollo tiene un coste de $(10 + 24.5) \cdot 18.25\mbox{\euro} = 629.63\mbox{\euro}$. 
El desarrollo de la subcontrata de la interfaz gráfica se factura por valor de $140\mbox{\euro}$.
El coste de la empresa, cuyos conceptos desglosamos en la \hyperref[sec:estimación_de_costes]{sección 4.1},
no podemos revelarlo.
Los importes son antes de impuestos.

\section{Trabajo futuro}

Una vez implementado el primer PMV analizamos las consecuencias de
nuestro producto. Probablemente será interesante en un futuro:

\begin{itemize}[noitemsep,nolistsep]
    \item \textbf{Proteger el procesamiento de datos ante ataques}. Ya que
          la información que obtenemos a través del bus CAN no
          tiene que superar ningún filtro un agente podría engañar
          los registros del sistema.
    \item \textbf{Añadir en la interfaz gráfica otra forma de acceder a los
          registros}. Por ejemplo, a través de un calendario o de una
          lista de registros.
\end{itemize}
