\chapter{Problema a resolver}

\textbf{Buscamos describir de una forma clara y concisa el problema y sus
circunstancias}. Consideramos que el problema se describe de forma clara y
concisa cuando se enuncia con el mínimo nivel de complejidad posible%
%
\footnote{Aunque
no por ello debe ser insuficiente. Un mito extendido acerca de las metodologías ágiles
es que no necesitamos tener una idea clara del problema antes de empezar a desarrollar
una solución. Nada mas lejos de la verdad, lo que se busca con los métodos ágiles en la
fase de análisis es poner a prueba empíricamente que nuestro modelo del problema es fiel
a la realidad, y que el problema que estamos resolviendo es el correcto antes de proponer
una solución arquitectónica. Así se concibe en las publicaciones que iniciaron el hoy
degenerado movimiento. [Cita Martin]
}%
, el lenguaje que se utiliza es simple pero adecuado
en su contexto y no tiene sesgos hacia ninguna solución.

\section{Análisis previo}

[ Explica cómo llevamos a cabo el primer análisis ]

\subsubsection{Conocimientos sobre los usuarios y el problema general (Comprender)}
% Forma parte del design thinking

Existen empresas y trabajadores los cuales, a cambio de dinero, trabajan la
tierra de un tercero con su maquinaria.

\textbf{Quien contrata el servicio} quiere reportes/resúmenes acerca de la actividad
de la máquina dado un intervalo de tiempo. Quiere conocer, a partir de una fecha y hora de inicio y otra de final:

\begin{enumerate}
   \item Tiempo medio y mediano entre vibraciones.
   \item Tiempo medio y mediano vibrando.
   \item Posición de los árboles vibrados.
   \item Tiempo de trabajo efectivo.
\end{enumerate}

\textbf{Tanto quien contrata el servicio como quien lo ejecuta} quieren tener una seguridad alta
en que los reportes/resúmenes son verdaderos y que no han sido modificados.

\textbf{El software tiene que poder ejecutarse en una plataforma hardware ya distribuida en el mercado.}
La maquinaria se controla mediante un sistema de control dedicado con las siguientes características:

\begin{itemize}
   \item Renesas RA6M1.
   \begin{itemize}
      \item Arm® Cortex®-M4.
      \item 8kB Data FLASH.
      \item 128kB SRAM.
      \item USB2.0 FS.
   \end{itemize}
   \item Conexión mediante UART a módulo de comunicaciones.
   \begin{itemize}
      \item BMD-380. (Sin firmware).
      \item ESP32-C3. (Firmware desarrollado: comunicación emulando UART a través de
      BLE).
      \item Digi XBee Pro. (Firmware de fábrica: UART a través de ZigBee.)
   \end{itemize}
   \item FRAM Fujitsu 4kiB. (Posibilidad de otros tamaños de almacenamiento). Driver
   desarrollado.
   \item Posibilidad de incluir otra FRAM Fujitsu o FLASH que sea compatible con la
   interfaz del integrado.
   \item Smart GNSS antenna module UBLOX CAM-M8Q. Driver desarrollado.
   \item Entradas 4-20mA. Driver desarrollado.
   \item Entradas digitales. Driver desarrollado.
   \item Entradas CAN 2.0. Driver desarrollado.
   \item Alimentación a 12V.
\end{itemize}

\subsubsection{Empatía con los usuarios mirándoles de cerca (Observar)}

Cualquier solución que se proponga debería de incluir interfaces
descriptivas y amigables para personas sin conocimientos de
electricidad o electrónica.

\subsubsection{Usuario típico (Definir el punto de vista)}

Empresa o trabajador que se dedica a tratar los árboles de campos a un tercero.

Suelen ser hombres que están acostumbrados al trabajo físico. Operan maquinaria
agrícola. Para comunicarse telemáticamente suelen utilizar Whatsapp y e-mail.

[ **Pendiente - Investigar hasta qué punto están dispuestos a voluntariamente
echar a andar la aplicación cada vez que quieran registrar un trabajo** ]

\subsection{Historias de usuario}

\subsection{Historias de desarrollador}

\subsubsection{HD??: feedback estimaciones}
Para mejorar la calidad de futuras estimaciones quiero registrar datos acerca del desarrollo
para poder:

\begin{itemize}
   \item Comparar las estimaciones de tiempo del segundo capítulo con los datos reales.
   \item Comparar las estimaciones de tiempo del segundo capítulo con los datos reales.
\end{itemize}

\subsection{\textit{User Journeys}}

\subsection{Síntesis}

[ Requisitos Funcionales ]

[ Requisitos No funcionales ]

\section{Análisis continuo}

\subsection{Historias de usuario}

\subsection{Historias de desarrollador}

\subsection{\textit{User Journeys}}

\subsection{Síntesis}

[ Requisitos Funcionales ]

[ Requisitos No funcionales ]
