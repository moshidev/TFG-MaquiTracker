\chapter{Arquitectura del sistema}

\section{Posibles soluciones}

\subsection{Generar todas las ideas posibles (Idear)}

\subsubsection{Histórico de trabajo (almacenamiento en control electrónico)}

Aunque podría ser más lento de recuperar y de desarrollar potencialmente
es más conveniente para el cliente.

\begin{enumerate}
   \item Almacenamiento de los datos.
   \begin{itemize}
       \item Para almacenar se puede utilizar \href{https://github.com/armink/FlashDB}{FlashDB}. 
       Único proyecto vivo profesional que encuentro apto para plataformas baremetal, listo para producción y de código abierto. Está mantenido activamente por una empresa. Permite TSDB y KVDB. Si vemos que no es suficiente al ser de código abierto podemos hacer una librería que dependa en el sistema o un fork con las mejoras en el software y un pull request de vuelta.
   \end{itemize}

   \item Generación de los datos de salida (con garantías).
   \begin{itemize}
       \item Se puede generar un PDF en la tarjeta y firmarlo.
       \item Se puede generar un bloque de texto en la tarjeta y firmarlo.
       \item Se puede generar una estructura binaria y firmarla.
   \end{itemize}

   \item Transmisión de los datos a un \textit{smartphone}, ordenador o similar.
   \begin{itemize}
       \item Mediante UART-BLE. \href{https://stackoverflow.com/a/22919464}{5kB/s~10kB/s}.
       \begin{itemize}
           \item Módulo piggyback ESP-C3 o BDM-380.
           \item Requiere que al otro lado se ejecute un software nuestro.
       \end{itemize}
       \item Mediante WiFi Direct.
       \begin{itemize}
           \item Diseñando un módulo piggyback con algún módulo compatible con este protocolo.
           \item No es compatible con iPhone, al menos no de una forma fácil.
           \begin{itemize}
               \item Requiere una aplicación.
           \end{itemize}
           \item No sabemos el porcentaje de móviles Android que lo implementan.
       \end{itemize}
       \item Mediante WiFi.
       \begin{itemize}
           \item Diseñando un módulo piggyback con algún módulo compatible con este protocolo.
           \begin{itemize}
               \item Requiere implementar un servidor SFTP, HTTP, FTP o similar.
           \end{itemize}
           \item Requiere que al otro lado se ejecute un software nuestro.
       \end{itemize}
       \item Mediante Bluetooth File Transfer profile.
       \begin{itemize}
           \item Diseñando un módulo piggyback con algún módulo compatible con este protocolo.
           \item No es compatible con iPhone.
       \end{itemize}
       \item Mediante Bluetooth Personal Area Network profile.
       \begin{itemize}
           \item Diseñando un módulo piggyback con algún módulo compatible con este protocolo.
           \item Es compatible con el iPhone sin necesidad de hacerse parte del programa MFi.
           \item Requiere que al otro lado se ejecute un software nuestro.
       \end{itemize}
       \item Mediante USB.
       \begin{itemize}
           \item Requiere utilizar algún stack que nos permita utilizar el protocolo USB.
           \item Requiere utilizar algún stack que nos permita interactuar con un sistema de ficheros FAT o similar.
           \item Requiere adaptar físicamente la placa y la caja, garantizando que este agujero no permita que entre polvo al control.
       \end{itemize}
   \end{itemize}
\end{enumerate}

\subsubsection{Histórico de trabajo (almacenamiento en *smartphone*)}

Aunque requeriría que el cliente estuviese atento de iniciar y finalizar
una sesión de trabajo desde su teléfono, potencialmente puede ser más
rápido y barato de desarrollar.

Parte de la premisa de que lo que se quiere extraer de aquí es la información
de un día de trabajo y no el histórico de una máquina.

\begin{enumerate}
    \item Transmisión de datos entre control y *smartphone*. (con garantías)
    \begin{itemize}
        \item BLE Peripheral Publisher.
        \item ¿Utilizando el procedimiento \href{https://lpccs-docs.renesas.com/Tutorial-DA145x-BLE-Security/access_and_signing.html}{Connection Data Signing}.?
        \item Necesitamos una aplicación de móvil que se suscriba a una característica.
        \item En iOS el servicio parece funcionar en primer plano y en segundo plano
              te suscribes a una característica en concreto.
        \item En Android va a ser más complicado de testar, pero en principio
              rece más o menos estable.
        \item Necesitamos tener un buffer en el control para el caso de que se
    desconecte el dispositivo, así como un aviso por pantalla de que se
    ha perdido la conexión.
    \end{itemize}
    \item Almacenamiento de datos. (con garantías)
    \begin{itemize}
        \item Firestore/Realtime Database. Implementado Android e iOS,
        opcionalmente puede sincronizarse con una base de datos en
        Google Cloud. El framework de Firebase ofrece otros servicios.
        De pago.
        \item SQLite. Gestor de bases de datos SQL. Amplia adopción y de código
        abierto.
        \item MongoDB. Gestor de bases de datos NoSQL. Amplia adopción y de código
        abierto.
    \end{itemize}
\end{enumerate}

\subsubsection{Histórico de trabajo (visualización de los datos)}

[ REPLANTEAR COMPLETAMENTE Y AÑADIR MÁS OPCIONES MÁS ESPECÍFICAS ]

1. Comprobación de la validez de los datos y visualización.
    - Mediante un servicio web que permita visualizar los datos en la
    propia página web.
        - Requeriría un mantenimiento durante muchos años por nuestra parte.
    - Mediante un servicio web que permita transformar los datos a un PDF.
        - Requeriría un mantenimiento durante muchos años por nuestra parte.
    - Mediante un software que permita transformar los datos a un PDF.
        - Requeriría distintos ports a distintas plataformas.
    - Mediante un software que permita visualizar y consultar los datos.
        - Requeriría distintos ports a distintas plataformas.
    - Mezcla de alguna de las anteriores.
