\chapter{Introducción}

Este Trabajo de Fin de Grado consiste en construir una solución software
capaz de emitir, monitorizar y registrar una serie de eventos en un
sistema de control hidráulico embebido. 

Este proyecto es software libre, y está liberado con
la licencia \textbf{GNU Affero General Public License} \cite{agplv3}.

\section{Motivación}

En España el mercado de los cultivos leñosos supone un ??? del PIB
anual, siendo un sector que emplea a más de ??? personas. [cita] Entre estos
predominan el cultivo de olivos, almendros, pistachos, cítricos y
viñedos. [cita] 

La explotación de estas plantaciones se realiza mayoritariamente por personas
físicas y no por empresas. [cita] La profesionalización de los propietarios
es creciente, pero baja. [cita] A su vez ocurre que no existe una renovación
de los jefes de explotación, con un porcentaje de gestores menores de 45 años
inferior al 40\%. La formación reglada específica de los jefes en el caso de
las explotaciones agrarias es inferior al 25\% en todas las comarcas españolas,
llegando a menos del ??? en aquellas regiones con mayor
explotación de este tipo de cultivos. [cita] Encontramos la mayor superficie
de cultivos leñosos en la mitad sur de la península, destacando Andalucía,
Castilla la Mancha y Extremadura. [cita INE https://storymaps.arcgis.com/stories/6fa8c26ecbde4fd99a0446e874c64898]

Desgraciadamente, el bajo nivel de automatización y de inversión por parte
de los propietarios de las explotaciones agrarias hace que la rentabilidad
de estos haya disminuido en los últimos años. [cita]
El sector de la
mecanización de este tipo de cultivos ha crecido en los últimos años de una
forma lenta y con algunas pausas, dado el alto coste de la maquinaria, la
falta de emprendimiento en el sector y el hecho de que la mayoría de las
explotaciones son de pequeño tamaño. [cita] 

Dos terceras partes del
coste de este tipo de cultivos van destinados a la recolección y a la poda.
[cita https://centroliva.com/mecanizacion-de-la-recoleccion-del-olivar-desde-la-investigacion-hasta-la-industria/]
Entre las soluciones mecanizadas
para la recolección encontramos principalmente los vibradores y los paraguas,
mientras que para la poda encontramos las tijeras eléctricas, las sierras y
las trituradoras. [cita]

Son muchos los jefes de explotación que deciden externalizar la recolección
de sus cultivos a empresas especializadas. [cita?] Ocurre un problema en este
caso, y es que el jefe de explotación deja de tener la certeza de que la
recolección se ha realizado de forma correcta, y de que la cantidad de producto
recolectado es la que se espera. Este problema sucede también en otros
contextos, como el de grandes fincas profesionalizadas.

Dada la desconfianza de muchos de los participantes del sector a la investigación y
el desarrollo de nuevas formas de producción y maquinaria agrícola creo que
es importante demostrar la efectividad de las soluciones por medio de los hechos.
[ a lo mejor sobra, o debería ir en otro sitio ]

\section{Objetivo}

Este proyecto busca desarrollar una solución software que permita
registrar, exportar y visualizar información sobre la recolección de cultivos
leñosos con paraguas vibradores según el criterio de nuestro cliente,
con la restricción de que pueda ejecutarse en una plataforma
hardware ya distribuida en el mercado.

Entre otros objetivos secundarios se encuentran:

\begin{enumerate}
    \item Automatización de los procesos de desarrollo y despliegue de la solución, como es
    la generación de documentación, la ejecución de pruebas, la generación de librerías
    y el despliegue en la plataforma hardware.
    \item Escribir código que cumpla los estándares de la industria para maquinaria pesada,
    asegurando que no puedan ocurrir situaciones inesperadas en tiempo de ejecución que puedan
    comprometer la integridad del código de control, que se ejecutará sobre la misma plataforma
    hardware.
    \item Generar una documentación adecuada para que cualquier desarrollador pueda entender,
    mantener y ampliar este proyecto en el futuro.
\end{enumerate}

\section{Estructura del documento}

[ Explica qué secciones tiene el documento y qué desarrollan ]

[ Explica el por qué de la elección ]
