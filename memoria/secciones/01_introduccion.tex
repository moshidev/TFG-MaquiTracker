\chapter{Introducción}

Este Trabajo de Fin de Grado consiste en construir una solución software
capaz de registrar, exportar y visualizar información sobre la recolección de cultivos
leñosos haciendo uso de un sistema de control embebido.

Este proyecto es software libre, y está liberado con
la licencia \textbf{GNU Affero General Public License} \cite{agplv3}.

\section{Motivación}

[ BORRADOR ]

En España el mercado de los cultivos leñosos supone un ??? del PIB
anual, siendo un sector que emplea a más de ??? personas. [cita] Entre estos
predominan el cultivo de olivos, almendros, pistachos, cítricos y
viñedos. [cita] 

La explotación de estas plantaciones se realiza mayoritariamente por personas
físicas y no por empresas. [cita] La profesionalización de los propietarios
es creciente, pero baja. [cita] A su vez ocurre que no existe una renovación
de los jefes de explotación, con un porcentaje de gestores menores de 45 años
inferior al 40\%. La formación reglada específica de los jefes en el caso de
las explotaciones agrarias es inferior al 25\% en todas las comarcas españolas,
llegando a menos del ??? en aquellas regiones con mayor
explotación de este tipo de cultivos. [cita] Encontramos la mayor superficie
de cultivos leñosos en la mitad sur de la península, destacando Andalucía,
Castilla la Mancha y Extremadura. [cita INE https://storymaps.arcgis.com/stories/6fa8c26ecbde4fd99a0446e874c64898]

Desgraciadamente, el bajo nivel de automatización y de inversión por parte
de los propietarios de las explotaciones agrarias hace que la rentabilidad
de estos haya disminuido en los últimos años. [cita]
El sector de la
mecanización de este tipo de cultivos ha crecido en los últimos años de una
forma lenta y con algunas pausas, dado el alto coste de la maquinaria, la
falta de emprendimiento en el sector y el hecho de que la mayoría de las
explotaciones son de pequeño tamaño. [cita] 

Dos terceras partes del
coste de este tipo de cultivos van destinados a la recolección y a la poda.
[cita https://centroliva.com/mecanizacion-de-la-recoleccion-del-olivar-desde-la-investigacion-hasta-la-industria/]
Entre las soluciones mecanizadas
para la recolección encontramos principalmente los vibradores y los paraguas,
mientras que para la poda encontramos las tijeras eléctricas, las sierras y
las trituradoras. [cita]

[ Crear o extraer gráficos del INE que ayuden a entender los datos aquí expuestos ]

Son muchos los jefes de explotación que deciden externalizar la recolección
de sus cultivos a empresas especializadas. [cita?] Ocurre un problema en este
caso, y es que el jefe de explotación deja de tener la certeza de que la
recolección se ha realizado de forma correcta, y de que la cantidad de producto
recolectado es la que se espera. Este problema sucede también en otros
contextos, como el de grandes fincas profesionalizadas.


\section{Objetivo}

Este proyecto busca \textbf{desarrollar una solución software que permita
registrar, exportar y visualizar información sobre la recolección de cultivos
leñosos con paraguas vibradores según los criterios de nuestro cliente},
con la restricción de que el registro y el exportado pueda ejecutarse en
una plataforma hardware ya distribuida en el mercado.

Este objetivo se sustenta en otros sub-objetivos, entre otros:

\begin{enumerate}
    \item \textbf{Automatización de los procesos de desarrollo y despliegue de la solución}, como es
    la generación de documentación, la ejecución de pruebas, la generación de librerías
    y el despliegue en la plataforma hardware, \textbf{de forma que minimicemos el tiempo de feedback}
    una vez realicemos un cambio y \textbf{de forma que tengamos una única fuente de verdad} acerca de cómo
    validar y cómo desplegar.
    \item \textbf{Código que cumpla los estándares de la industria para maquinaria pesada},
    asegurando que no puedan ocurrir situaciones inesperadas en tiempo de ejecución que puedan
    comprometer la integridad del código de control, que se ejecutará sobre la misma plataforma
    hardware.
    \item \textbf{Documentación adecuada para que cualquier desarrollador pueda entender,
    mantener y ampliar este proyecto} en el futuro, de forma que maximicemos el atractivo
    del proyecto y de forma que minimicemos la frustración de un contribuyente potencial
    al proyecto de software libre.
\end{enumerate}

\section{Estructura del documento}

Esta memoria refleja el proceso de desarrollo de la solución software.
Cada una de las secciones expuestas representa una etapa del desarrollo.
\textbf{Etapas posteriores se basan en las anteriores, y pueden corregirlas.}

\begin{enumerate}
    \item \textbf{Introducción.} Donde exponemos el problema a resolver de forma general,
    sus circunstancias actuales y los objetivos que se quieren alcanzar en rasgos generales.

    \item \textbf{Problema a resolver.} Donde describimos el problema de una forma clara y concisa.
    Enunciamos qué se quiere resolver, quién va a interactuar con el programa y en qué circunstancias
    se interactuaría con el programa.

    \item \textbf{Arquitectura del sistema.} Donde enumeramos distintas soluciones plausibles
    y elegimos una de ellas junto al cliente.

    \item \textbf{Planificación.} Donde exponemos cómo abordamos el desarrollo del problema.
    Explicamos qué metodología seguimos, cómo estimamos el tiempo y los costes, cómo seguimos el progreso
    del proyecto y cómo registramos métricas para los objetivos del sexto capítulo de esta memoria.

    \item \textbf{Implementación.} Donde describimos cómo llevamos a cabo el desarrollo software
    de la solución propuesta.
    \begin{itemize}
        \item Escucha y registro de eventos.
        \item Almacenamiento en memoria persistente.
        \item Extracción de datos.
        \item Firma de los datos.
        \item Transmisión de los datos.
        \item Visualización de los datos.
    \end{itemize}

    \item \textbf{Comparación de los datos previos y posteriores al desarrollo.} Donde evaluamos cómo de
    acertados fueron los análisis del segundo, tercer y cuarto capítulo de esta memoria.

    \item \textbf{Conclusión.} Donde realizamos una visita crítica a las ideas de los puntos
    anteriores una vez finalizado el proyecto y exponemos nuevas ideas surgidas de la experiencia.
\end{enumerate}
