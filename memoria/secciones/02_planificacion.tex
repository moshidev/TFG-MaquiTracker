\chapter{Planificación}

\textbf{Buscamos describir cómo abordamos el desarrollo del problema y su solución}.
Leyendo este capítulo el lector conocerá la metodología que empleamos al desarrollar este
proyecto.

Es inmensamente doloroso el poco tiempo que vivimos, pero es que es más doloroso dedicar
nuestro tiempo a tareas que no benefician a aquellos entes a los cuales queremos aportar
valor.
\textbf{A lo largo de este Trabajo de Fin de Grado exponemos diversos métodos para minimizar
el tiempo invertido en tareas sin valor para nuestro propósito}.

% Objeto: Describir cómo planeamos organizar el trabajo, y qué categorías vamos a cubrir

% Categorías a cubrir:
% 1. Definición de objetivos.
% 2. Plan y método para llegar a los objetivos. (Trunk Based, Continuous Delivery, Design Thinking)
% 2.1 Tanto para implementación, como para memoria, como para obtención de métricas.
% 3. Estimación de costes

\section{Método para descubrir las tareas a realizar}

\section{Método para descubrir el problema a resolver}

Nos servimos de distintas herramientas para descubrir el conocimiento que necesitamos
para desarrollar un sistema informático: correcta identificación del problema a resolver,
mejor solución posible teniendo en cuenta el entorno en el que se utilizará,
y formulación de la solución de este en términos de datos objetivos acerca de cómo debe de
funcionar un sistema.

Formulamos el problema haciendo uso de las fases de empatizar y definir del método
\textit{Design Thinking} para después, tanto en un análisis previo como a medida
que vamos desarrollando el proyecto, extraer historias de usuario, historias de
desarrollador y \textit{user journeys} de donde obtenemos los objetivos deseados
junto a sus respectivas importancias.
Finalmente sintetizamos esta información en forma de requisitos funcionales y
requisitos no funcionales.

\subsection{Conversaciones con el cliente}

Este primer método lo ideamos una vez el cliente nos comunica brevemente su necesidad.
\textbf{Su principal objetivo es el de recuperar y modelar el problema a resolver}, ya sea en
el análisis previo o como complementario del análisis continuo, pero también busca
otros objetivos como el de transmitir confianza al cliente o el de ayudarnos como empresa a
decidir si esta será una batalla en la que nos gustaría e interesaría participar.

\textbf{Como primer paso tenemos que el cliente comunique su necesidad.} Sólo por el hecho de que esta
persona física o jurídica está depositando confianza en nosotros merece el más
adecuado respeto, que se demuestra dejándole el espacio que necesita para exponer
sus ideas. De esta descripción extraemos información verbal y
no verbal que nos ayuda a comprender las circunstancias en que ocurre el potencial
problema a resolver, y en caso de no ser un cliente de confianza o desconocer las
circunstancias en que se desenvuelve el problema, si como empresa nos interesa
ser un proveedor de soluciones para él y su sector.

De una forma atenta, pero crítica, escuchamos y anotamos la información que el
cliente quiere transmitirnos. Si existe algún término que enuncia y desconocemos
o alguna circunstancia que nos describe y que no llegamos a comprender o bien le
interrumpimos o bien anotamos para preguntar más adelante.

\textbf{El segundo paso es identificar la raíz del problema.} Por experiencia sabemos
que las personas tienden a enunciar los problemas en términos de la solución%
%
\footnote{[ Code Complete 2 ]
}%
. Muchas veces aquello que nos solicitan responde a una necesidad subyacente que no
se identifica o, si identificada, no se comunica.

Buscamos dialogando con el cliente y por medio de nuestra propia investigación exponer
el problema eliminando las restricciones innecesarias, buscando aquellas que son
necesarias y, a poder ser, sin omitir aspectos que pudiesen afectar a la arquitectura
de la solución.

Por ejemplo, proponemos borradores rápidos de arquitectura los cuales
nuestro cliente puede analizar de forma crítica, de forma independiente
o por medio del método dialéctico. El interesado desmiente aquellos aspectos
pertinentes y saca a la luz nuevas necesidades y restricciones que no se habían
comunicado.

[ Tercer paso: formalización de las conversaciones por medio de las herramientas.
Explica que lo hacemos por ejemplificar, pero que el uso de estas herramientas
dependerá en nuestros conocimientos del sector, en el tamaño y toma del contacto
del sector de nuestro equipo, en la complejidad del problema y en el nivel de intuición
que podamos tener en este ]

\subsection{Design Thinking}

%[ Explica en términos generales Design Thinking. ]
%[ Explica por qué la técnica del Design Thinking puede ayudarnos a este propósito ]
%[ Explica cómo adaptamos la técnica a nuestras necesidades. ]

\subsection{Historias de usuario}

%[ Enuncia problema que resuelven las historias de usuario ]
%[ Describe cómo resuelve el problema la historia de usuario ]
%[ Describe la esencia de una historia de usuario ]

\subsection{Historias de desarrollador}

%[ Bridge desde la historia de usuario hacia las historias de desarrollador ]

\subsection{\textit{User journeys}}

%[ Igual que con las historias de usuario pero para las user stories y
%los requisitos funcionales y no funcionales ]

\subsection{Requisitos funcionales}

\subsection{Requisitos no funcionales}


\section{Metodo para el desarrollo software}

[ Describe Trunk Based Development ]

[ Describe Continuous Delivery ]

[ Explica el por qué de la elección ]

[ ¿Anexo sobre la historia de las distintas metodologías y
filosofías de desarrollo de software? ]

Para llevar a cabo este producto software intentaremos seguir algunas
pautas que buscarán acortar nuestros tiempos de \textit{feedback}.

\subsection{¿Por qué acortar los tiempos de \textit{feedback}?}

Básicamente porque \textbf{cuanto antes detectamos que nos estamos equivocando
antes podremos corregirlo}, y antes podremos seguir adelante.

Pensemos en la situación de un enamorado sin experiencia que durante meses dedica su
tiempo, su energía y su creatividad a componerle una canción a una chica.
Durante este tiempo, este individuo mejora de forma objetiva la
melodía, letra y sensaciones que la canción genera en sí mismo
y en sus amistades. Ahora es un mejor compositor, tiene un mejor manejo
de los instrumentos que utiliza e incluso ha mejorado su
relación con sus amigos y su familia.

El chico piensa que así puede despertar o avivar un sentimiento recíproco
en la chica. Sin embargo, cuando se presenta la canción,
la realidad se impone. Entre otros muchos factores,
la falta de sincronización entre los sentimientos de ella y de él causan
de forma instintiva un rechazo en ella.

Esta analogía \textbf{la mostramos para ejemplificar lo absurdas e idealistas
que son estas situaciones que se dan en otros campos de nuestra vida.}
Productos excelentes desde un punto de vista técnico que se
congelan porque no se discutió un precio objetivo con
el cliente en una fase temprana del desarrollo, ideas felices
por parte de \textit{stakeholders} sobre qué debe de
ser capaz de hacer el software que terminan en años de desarrollo
y con objetivos sin cumplir, etc.

Si el chico le hubiese propuesto de quedar de una forma casual
habría llegado a conocerla mejor\footnote{¡O no! A lo mejor ella no
quiera quedar con él. Al chico le interesa
conocer esta realidad lo antes posible.}. Podría haberse dado cuenta que
ella no está acostumbrada a salir mucho de casa, que buscando sitios
interesantes con los que pasar tiempo con ella tal vez podría ganar su
confianza, y su cariño. Podría haber probado esta hipótesis, y cambiar
su estrategia según el \textit{feedback} que recibiría. Podría en este tiempo que
pasan juntos darse cuenta que no es la chica adecuada para él. Podría
incluso haberse dado cuenta que, la idea de la canción, podría ser buena
bajo unas determinadas circunstancias.

Los autores de \cite{accelerate} divulgan cómo afectan los tiempos de feedback
a los productos software. Nos centramos en los capítulos dedicados al impacto de
la entrega continua y al impacto de la entrega continua%
\footnote{%
Superconjunto de la integración continua. Básicamente consiste en la automatización
del proceso de desarrollo. \cite{ModernSoftwareEngineering} y \cite{minimumViableCD}
son buenos recursos para aprender más acerca de estos conceptos.%
}%
en la calidad. Según escriben, \textbf{existe una correlación estadística alta entre aquellos
equipos capaces de desplegar a producción a demanda y 
con mecanismos que ofrecen de forma clara un \textit{feedback} rápido en
la calidad y en la capacidad del software de desplegarse y entre el rendimiento del
equipo.}

El equipo DORA\footnote{DevOps Research and Assessment.} coincide en estos resultados.
En su publicación \cite{EliteDevOps} explican cuatro métricas que pueden medir el
rendimiento de una organización. Según explican en la entrada de blog citada,
la frecuencia de despliegue, el tiempo que tarda un commit desde que se sube
al servidor hasta que se lleva a producción,
el porcentaje de despliegues que causan un error en producción y el tiempo
que tarda una organización en recuperarse de un error son clave \textbf{para medir la
velocidad y la estabilidad en el desarrollo}.

\subsection{Productos para disminuir el tiempo de realimentación}

\begin{enumerate}
    \item \textbf{Automatización de los procesos de desarrollo y despliegue de la solución}, como es
    la generación de documentación, la ejecución de pruebas, la generación de librerías
    y el despliegue en la plataforma hardware, \textbf{de forma que minimicemos el tiempo de \textit{feedback}}
    una vez realicemos un cambio y \textbf{de forma que tengamos una única fuente de verdad} acerca de cómo
    validar y cómo desplegar.
\end{enumerate}

\subsection{Productos para mejorar la calidad del producto}

\begin{enumerate}
    \item \textbf{Pruebas software unitarias para cada funcionalidad implementada} que aseguren que no
    puedan ocurrir situaciones inesperadas en tiempo
    de ejecución que puedan comprometer la integridad del código de control, que se ejecutará sobre
    la misma plataforma hardware.
    \item \textbf{Documentación adecuada para que cualquier desarrollador pueda entender,
    mantener y ampliar este proyecto} en el futuro, de forma que maximicemos el atractivo
    del proyecto y de forma que minimicemos la frustración de un contribuyente potencial
    al proyecto de software libre.
\end{enumerate}

\section{Temporización}

[ Describe una planificación del tiempo de desarrollo del proyecto ]

[ Explica el por qué de la elección ]

\section{Seguimiento del desarrollo}

[ Describe la plantilla de costes no recurrentes asociada al proyecto ]

[ Explica el por qué de la elección ]
