\chapter{Planificación}

\section{Metodología utilizada}

[ Describe Trunk Based Development ]

[ Describe Continuous Delivery ]

[ Explica el por qué de la elección ]

[ ¿Anexo sobre la historia de las distintas metodologías y
filosofías de desarrollo de software? ]

Para llevar a cabo este producto software intentaremos seguir algunas
pautas que buscarán acortar nuestros tiempos de \textit{feedback}.

\subsection{¿Por qué acortar los tiempos de \textit{feedback}?}

Básicamente porque \textbf{cuanto antes detectamos que nos estamos equivocando
antes podremos corregirlo}, y antes podremos seguir adelante.

Pensemos en la situación de un enamorado sin experiencia que durante meses dedica su
tiempo, su energía y su creatividad a componerle una canción a una chica.
Durante este tiempo, este individuo mejora de forma objetiva la
melodía, letra y sensaciones que la canción genera en sí mismo
y en sus amistades. Ahora es un mejor compositor, tiene un mejor manejo
de los instrumentos que utiliza e incluso ha mejorado su
relación con sus amigos y su familia.

El chico piensa que así puede despertar o avivar un sentimiento recíproco
en la chica. Sin embargo, cuando se presenta la canción,
la realidad se impone. Entre otros muchos factores,
la falta de sincronización entre los sentimientos de ella y de él causan
de forma instintiva un rechazo en ella.

Esta analogía \textbf{la mostramos para ejemplificar lo absurdas e idealistas
que son estas situaciones que se dan en otros campos de nuestra vida.}
Productos excelentes desde un punto de vista técnico que se
congelan porque no se discutió un precio objetivo con
el cliente en una fase temprana del desarrollo, ideas felices
por parte de \textit{stakeholders} sobre qué debe de
ser capaz de hacer el software que terminan en años de desarrollo
y con objetivos sin cumplir, etc.

Si el chico le hubiese propuesto de quedar de una forma casual
habría llegado a conocerla mejor\footnote{¡O no! A lo mejor ella no
quiera quedar con él. Al chico le interesa
conocer esta realidad lo antes posible.}. Podría haberse dado cuenta que
ella no está acostumbrada a salir mucho de casa, que buscando sitios
interesantes con los que pasar tiempo con ella tal vez podría ganar su
confianza, y su cariño. Podría haber probado esta hipótesis, y cambiar
su estrategia según el \textit{feedback} que recibiría. Podría en este tiempo que
pasan juntos darse cuenta que no es la chica adecuada para él. Podría
incluso haberse dado cuenta que, la idea de la canción, podría ser buena
bajo unas determinadas circunstancias.

Los autores de \cite{accelerate} divulgan cómo afectan los tiempos de feedback
a los productos software. Nos centramos en los capítulos dedicados al impacto de
la entrega continua y al impacto de la entrega continua%
\footnote{%
Superconjunto de la integración continua. Básicamente consiste en la automatización
del proceso de desarrollo. \cite{ModernSoftwareEngineering} y \cite{minimumViableCD}
son buenos recursos para aprender más acerca de estos conceptos.%
}%
en la calidad. Según escriben, \textbf{existe una correlación estadística alta entre aquellos
equipos capaces de desplegar a producción a demanda y 
con mecanismos que ofrecen de forma clara un \textit{feedback} rápido en
la calidad y en la capacidad del software de desplegarse y entre el rendimiento del
equipo.}

El equipo DORA\footnote{DevOps Research and Assessment.} coincide en estos resultados.
En su publicación \cite{EliteDevOps} explican cuatro métricas que pueden medir el
rendimiento de una organización. Según explican en la entrada de blog citada,
la frecuencia de despliegue, el tiempo que tarda un commit desde que se sube
al servidor hasta que se lleva a producción,
el porcentaje de despliegues que causan un error en producción y el tiempo
que tarda una organización en recuperarse de un error son clave \textbf{para medir la
velocidad y la estabilidad en el desarrollo}.

\subsection{Productos para disminuir el tiempo de realimentación}

\begin{enumerate}
    \item \textbf{Automatización de los procesos de desarrollo y despliegue de la solución}, como es
    la generación de documentación, la ejecución de pruebas, la generación de librerías
    y el despliegue en la plataforma hardware, \textbf{de forma que minimicemos el tiempo de \textit{feedback}}
    una vez realicemos un cambio y \textbf{de forma que tengamos una única fuente de verdad} acerca de cómo
    validar y cómo desplegar.
\end{enumerate}

\subsection{Productos para mejorar la calidad del producto}

\begin{enumerate}
    \item \textbf{Pruebas software unitarias para cada funcionalidad implementada} que aseguren que no
    puedan ocurrir situaciones inesperadas en tiempo
    de ejecución que puedan comprometer la integridad del código de control, que se ejecutará sobre
    la misma plataforma hardware.
    \item \textbf{Documentación adecuada para que cualquier desarrollador pueda entender,
    mantener y ampliar este proyecto} en el futuro, de forma que maximicemos el atractivo
    del proyecto y de forma que minimicemos la frustración de un contribuyente potencial
    al proyecto de software libre.
\end{enumerate}

\section{Temporización}

[ Describe una planificación del tiempo de desarrollo del proyecto ]

[ Explica el por qué de la elección ]

\section{Seguimiento del desarrollo}

[ Describe la plantilla de costes no recurrentes asociada al proyecto ]

[ Explica el por qué de la elección ]
