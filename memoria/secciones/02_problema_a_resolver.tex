\chapter{Problema a resolver}

\textbf{Buscamos describir de una forma clara y concisa el problema y sus
circunstancias}. Consideramos que el problema se describe de forma clara y
concisa cuando se enuncia con el mínimo nivel de complejidad posible%
%
\footnote{Aunque
no por ello debe ser insuficiente. Un mito extendido acerca de las metodologías ágiles
es que no necesitamos tener una idea clara del problema antes de empezar a desarrollar
una solución. Nada mas lejos de la verdad, lo que se busca con los métodos ágiles en la
fase de análisis es poner a prueba empíricamente que nuestro modelo del problema es fiel
a la realidad, y que el problema que estamos resolviendo es el correcto antes de proponer
una solución arquitectónica. Así se concibe en las publicaciones que iniciaron el hoy
degenerado movimiento. [Cita Martin]
}%
, el lenguaje que se utiliza es simple pero adecuado
en su contexto y no tiene sesgos hacia ninguna solución. 

Formulamos el problema haciendo uso de las fases de empatizar y definir del método
\textit{Design Thinking} para después, tanto en un análisis previo como a medida
que vamos desarrollando el proyecto, extraer historias de usuario, historias de
desarrollador y \textit{user journeys} de donde obtenemos los objetivos deseados
junto a sus respectivas importancias.
Finalmente sintetizamos esta información en forma de requisitos funcionales y
requisitos no funcionales.

\section{Método}

Es inmensamente doloroso el poco tiempo que vivimos, pero es que es más doloroso dedicar
nuestro tiempo a tareas que no benefician a aquellos entes a los cuales queremos aportar
valor.
\textbf{A lo largo de este Trabajo de Fin de Grado exponemos diversos métodos para recibir
realimentación y para conseguirla en el menor tiempo posible}, de forma que minimicemos
el tiempo invertido en tareas sin valor para nuestro propósito.

Este primer método lo ideamos una vez el cliente nos comunica brevemente su necesidad.
\textbf{Su principal objetivo es el de recuperar y modelar el problema a resolver}, ya sea en
el análisis previo o como complementario del análisis continuo, pero también busca
otros objetivos como el de transmitir confianza al cliente o el de ayudarnos como empresa a
decidir si esta será una batalla en la que nos gustaría e interesaría participar.

\textbf{Como primer paso tenemos que el cliente comunique su necesidad.} Sólo por el hecho de que esta
persona física o jurídica está depositando confianza en nosotros merece el más
adecuado respeto, que se demuestra dejándole el espacio que necesita para exponer
sus ideas. De esta descripción extraemos información verbal y
no verbal que nos ayuda a comprender las circunstancias en que ocurre el potencial
problema a resolver, y en caso de no ser un cliente de confianza o desconocer las
circunstancias en que se desenvuelve el problema, si como empresa nos interesa
ser un proveedor de soluciones para él y su sector.

De una forma atenta, pero crítica, escuchamos y anotamos la información que el
cliente quiere transmitirnos. Si existe algún término que enuncia y desconocemos
o alguna circunstancia que nos describe y que no llegamos a comprender o bien le
interrumpimos o bien anotamos para preguntar más adelante.

\textbf{El segundo paso es identificar la raíz del problema.} Por experiencia sabemos
que las personas tienden a enunciar los problemas en términos de la solución%
%
\footnote{[ Code Complete 2 ]
}%
. Muchas veces aquello que nos solicitan responde a una necesidad subyacente que no
se identifica o, si identificada, no se comunica.

Buscamos dialogando con el cliente y por medio de nuestra propia investigación exponer
el problema eliminando las restricciones innecesarias, buscando aquellas que son
necesarias y, a poder ser, sin omitir aspectos que pudiesen afectar a la arquitectura
de la solución.

Por ejemplo, proponemos borradores rápidos de arquitectura los cuales
nuestro cliente puede analizar de forma crítica, de forma independiente
o por medio del método dialéctico. El interesado desmiente aquellos aspectos
pertinentes y saca a la luz nuevas necesidades y restricciones que no se habían
comunicado.

[ Tercer paso: formalización de las conversaciones por medio de las herramientas.
Explica que lo hacemos por ejemplificar, pero que el uso de estas herramientas
dependerá en nuestros conocimientos del sector, en el tamaño y toma del contacto
del sector de nuestro equipo, en la complejidad del problema y en el nivel de intuición
que podamos tener en este ]

\subsection{Herramientas}

[ Enuncia problema que resuelven las historias de usuario ]

[ Describe cómo resuelve el problema la historia de usuario ]

[ Describe la esencia de una historia de usuario ]

[ Bridge desde la historia de usuario hacia las historias de desarrollador ]

[ Igual que con las historias de usuario pero para las user stories y
los requisitos funcionales y no funcionales ]

\section{Análisis previo}

[ Explica los objetivos del análisis previo ]

\subsection{Enunciado del problema haciendo uso de la técnica del \textit{Design Thinking}}

[ Explica en términos generales Design Thinking. ]

[ Explica por qué la técnica del Design Thinking puede ayudarnos a este propósito ]

[ Explica cómo adaptamos la técnica a nuestras necesidades. ]

\subsubsection{Conocimientos sobre los usuarios y el problema general (Comprender)}

Existen empresas y trabajadores los cuales, a cambio de dinero, trabajan la
tierra de un tercero con su maquinaria.

\textbf{Quien contrata el servicio} quiere reportes/resúmenes acerca de la actividad
de la máquina dado un intervalo de tiempo. Quiere conocer, a partir de una fecha y hora de inicio y otra de final:

\begin{enumerate}
   \item Tiempo medio y mediano entre vibraciones.
   \item Tiempo medio y mediano vibrando.
   \item Posición de los árboles vibrados.
   \item Tiempo de trabajo efectivo.
\end{enumerate}

\textbf{Tanto quien contrata el servicio como quien lo ejecuta} quieren tener una seguridad alta
en que los reportes/resúmenes son verdaderos y que no han sido modificados.

\textbf{El software tiene que poder ejecutarse en una plataforma hardware ya distribuida en el mercado.}
La maquinaria se controla mediante un sistema de control dedicado con las siguientes características:

\begin{itemize}
   \item Renesas RA6M1.
   \begin{itemize}
      \item Arm® Cortex®-M4.
      \item 8kB Data FLASH.
      \item 128kB SRAM.
      \item USB2.0 FS.
   \end{itemize}
   \item Conexión mediante UART a módulo de comunicaciones.
   \begin{itemize}
      \item BMD-380. (Sin firmware).
      \item ESP32-C3. (Firmware desarrollado: comunicación emulando UART a través de
      BLE).
      \item Digi XBee Pro. (Firmware de fábrica: UART a través de ZigBee.)
   \end{itemize}
   \item FRAM Fujitsu 4kiB. (Posibilidad de otros tamaños de almacenamiento). Driver
   desarrollado.
   \item Posibilidad de incluir otra FRAM Fujitsu o FLASH que sea compatible con la
   interfaz del integrado.
   \item Smart GNSS antenna module UBLOX CAM-M8Q. Driver desarrollado.
   \item Entradas 4-20mA. Driver desarrollado.
   \item Entradas digitales. Driver desarrollado.
   \item Entradas CAN 2.0. Driver desarrollado.
   \item Alimentación a 12V.
\end{itemize}

\subsubsection{Empatía con los usuarios mirándoles de cerca (Observar)}

Cualquier solución que se proponga debería de incluir interfaces
descriptivas y amigables para personas sin conocimientos de
electricidad o electrónica.

\subsubsection{Usuario típico (Definir el punto de vista)}

Empresa o trabajador que se dedica a tratar los árboles de campos a un tercero.

Suelen ser hombres que están acostumbrados al trabajo físico. Operan maquinaria
agrícola. Para comunicarse telemáticamente suelen utilizar Whatsapp y e-mail.

[ **Pendiente - Investigar hasta qué punto están dispuestos a voluntariamente
echar a andar la aplicación cada vez que quieran registrar un trabajo** ]

\subsection{Historias de usuario}

\subsection{Historias de desarrollador}

\subsubsection{HD??: feedback estimaciones}
Para mejorar la calidad de futuras estimaciones quiero registrar datos acerca del desarrollo
para poder:

\begin{itemize}
   \item Comparar las estimaciones de tiempo del cuarto capítulo con los datos reales.
   \item Comparar las estimaciones de tiempo del cuarto capítulo con los datos reales.
\end{itemize}

\subsection{\textit{User Journeys}}

\subsection{Síntesis}

[ Requisitos Funcionales ]

[ Requisitos No funcionales ]

\section{Análisis continuo}

\subsection{Historias de usuario}

\subsection{Historias de desarrollador}

\subsection{\textit{User Journeys}}

\subsection{Síntesis}

[ Requisitos Funcionales ]

[ Requisitos No funcionales ]
