\thispagestyle{empty}

\begin{center}
{\large\bfseries \titulo \\ \subtitulo }\\
\end{center}
\begin{center}
\titulo \\
\end{center}

%\vspace{0.7cm}

\vspace{0.5cm}
\noindent\textbf{Palabras clave}: \textit{software libre, TDD, automoción, monitorización, cultivos leñosos, paraguas vibrador}
\vspace{0.7cm}

\noindent\textbf{Resumen}\\

La recolecta mecanizada del fruto de cultivos leñosos se lleva a cabo en España mayoritariamente
con paraguas vibradores. Actualmente no existe ninguna solución económica integrada en el control
que permita extraer datos acerca del uso de la máquina dado un día.

Exponemos una metodología de desarrollo para proyectos de esta complejidad, tanto a nivel de problema
como a nivel de número de implicados, y desarrollamos una solución a desplegar en máquinas piloto.

\cleardoublepage

\begin{center}
	{\large\bfseries \tituloingles \\ \subtituloingles}\\
\end{center}
\begin{center}
	\autor\\
\end{center}
\vspace{0.5cm}
\noindent\textbf{Keywords}: \textit{open source, TDD, automotive, monitoring, woody crops, tree shaker}
\vspace{0.7cm}

\noindent\textbf{Abstract}\\

In Spain, mechanized harvesting of woody crops like olive and almond trees are mostly carried out with \textit{tree shakers}.
There is currently no cost-effective solution integrated in the control system
that allows to extract data about the use of the machine in a given day.

We present a development methodology for projects of this complexity, both in terms of the problem and the number of people involved.
We develop a solution to be deployed on pilot machines.

\cleardoublepage

\thispagestyle{empty}

\noindent\rule[-1ex]{\textwidth}{2pt}\\[4.5ex]

D. \textbf{\tutor}, Profesor(a) del \departamento.

\vspace{0.5cm}

\textbf{Informo:}

\vspace{0.5cm}

Que el presente trabajo, titulado \textit{\textbf{\titulo}},
ha sido realizado bajo mi supervisión por \textbf{\autor}, y autorizo la defensa de dicho trabajo ante el tribunal
que corresponda.

\vspace{0.5cm}

Y para que conste, expiden y firman el presente informe en Granada a septiembre del 2024.

\vspace{1cm}

\textbf{El/la director(a)/es: }

\vspace{5cm}

\noindent \textbf{\tutor}

\chapter*{Dedicatoria personal}

\begin{center}
	\textit{A mis padres y a mi hermana,}

	\textit{por su apoyo y amor incondicional.}
\end{center}


